\documentclass{article}
\usepackage{geometry}
\usepackage{fancyhdr}
\usepackage{amsmath ,amsthm ,amssymb}
\usepackage{graphicx}
\usepackage{hyperref}

\begin{document}

The word from on high...

\begin{quote}
...the development of new mathematical approaches for an integrated computational and experimental pipeline for systematic improvment of biomolecular energy functions and protein design methodology...
\end{quote}

\begin{quote}
...The design process consists of choosing a new protein structure or function (binding, catalysis, etc), computing sequences predicted to have as their lowest energy state the desired structure and/or function, and then manufacturing and testing the designs...
\end{quote}

\begin{quote}
...Third, in the design computations, how can the energy of the designed systems most effectively be minimized so that the feedback is on actual ground states or near ground states...
\end{quote}


TO ADDRESS: search and diversity

\section{Protein Design Methodology}

\subsection{Fitness}\label{fitness}

The goal of protein design is to produce a sequence of amino acids that will
fold into a protein that performs a desired function. The protein designer
defines a function $Fitness(S,X)$ mapping a sequence $S$ along with a three
dimensional structure X to a scalar quantity assessing fitness for a desired
purpose. We attempt to optimize $Fitness(S,X)$ subject to the constraint that
sequence $S$ must actually fold into structure X. In order to define reasonable
fitness functions and assess the folding of S, we employ a physical model of
protein energetics to define an energy function $Energy(S,X)$ mapping a
sequence-structure pair to an estimate of the free energy of the system. Desired
fitness is often closely related to energy. In the simple case of optimizing for
stability, $Energy(S,X)$ may be used directly as a fitness function. To design a
protein P that will bind a target T, an appropriate fitness function could be
the energy delta upon binding: $Fitness(S,X) = Energy(S,XP) + Energy(XT) -
Energy(S,X)$, where $S$ is the sequence of P and X is a combined structure of P docked to
T. In addition to optimizing fitness for purpose, we must also consider that
sequence $S$ may not fold into the desired structure X. Thus, we must optimize
$Fitness(S,X)$ subject to the constraint that X is the minimum of $Energy(S,X)$
given S.

\subsection{Independent Variables}

Protein design is technically challenging due to both scale and a mixed
continuous/discrete character of the involved problems. Given an alphabet of the
twenty natural amino-acids, the number of possible sequences $S$ is $20^N$. The
space of possible structures X is comparably large but consisting of continuous
degrees of freedom, comprising several types with different properties:
amino acid backbone ($\Phi^{3N}$), amino acid side-chain ($\chi^{\approx2N}$),
and rigid-body ($\Re^6$).
$\Phi^{3N}$ arise from the homogeneous backbone portion of amino acid residues
containing three rotate-able bonds common across most amino acids. $\chi^{\approx2N}$
arise from the heterogeneous portion of amino acid residues, the contain up to
four rotatable chi-angles per residue with diverse structural characteristics.
$\Re^6$ arise from interactions between kinematically disconnected bodies that
could be protein chains, small molecules, nanoparticles, surfaces, et cetera. $\Phi^{3N}$
and $\Re^6$ are independent of sequence, while $\chi^{\approx2N}$ are only well-defined
for a particular sequence.
A typical calculation designing a small 60 amino acid
protein to bind a target involves considering a 180 dimensional space of $\Phi^{3N}$
combined with a six dimensional rigid body space, each with $10^78$ possible
sequences, each with a roughly 120 dimensional spaces $\chi^{\approx2N}$.
The scope of
this problem space is daunting (REF Levinthal paradox), but allowable
configurations are highly constrained. $\Phi^{3N}$ are stereotyped by the formation
of alpha helix and beta sheet secondary structures, and the tendency for
adjacent sections of the protein chain to form locally optimal substructures.
Global sampling of $\Phi^{3N}$ is preformed in coordinated fashion using short
structural fragments harvested from structures of natural proteins (REF
fragments). $\chi^{\approx2N}$ are stereotyped by the detailed structure of their
respective amino acid side chains, which occur in natural proteins within
discrete energy wells known as rotamers (REF dunbrack). Global optimization of
CHI space can thus be incorporated into discrete sequence optimization by
expanding the amino acid alphabet to include multiple letters per amino acid,
one per rotameric energy well. Unlike $\Phi^{3N}$ and CHI, $\Re^6$ are not stereotyped in
the same manner as $\Phi^{3N}$ and $\chi^{\approx2N}$ both because the extent of each dimension is
higher, and because the rigid body relationships they capture are directly
intermediated by diverse side chain interactions. Thus, $\Re^6$ contribute
disproportionately more complexity to the protein design problem than a naive
accounting of dimensionality would indicate.

\subsection{Phases of Search/Optimization}

Current optimization methodology for protein design makes extensive use of monte
carlo search and optimization in various forms along with continuous
minimization. Each candidate sequence-structure pair is generated in three
phases.

Phase 1, $\Phi^{3N}$ and $\Re^6$ are optimized using a smoothed, sequence
independent version of the fitness function. For de novo design of protein folds
we focus on $\Phi^{3N}$, using monte carlo sampling of local structural fragments is
used. For binder/enzyme design, $\Phi^{3N}$ are set from predefined scaffold
proteins, and $\Re^6$ can be handled with a variety of docking technologies
(section \ref{proposed1})

Phase 2, atomic detail is added by combinatorially optimizing
sequence and $\chi^{\approx2N}$ using monte carlo or in conjunction with a
fitness function (section \ref{fitness}),
given $\Phi^{3N}$ and $\Re^6$ positions from Phase 1.

Phase 3, local continuous minimization of all continuous $\Phi^{3N}$, $\Re^6$,
and $\chi^{\approx2N}$ is performed using [LBFGS or similar techniques that can
handle high dimensional problems, taking advantage
of gradient information available from our fitness functions and physical model,
? REF?].

Phases 2 and 3 may be cycled several times in a single trajectory. As
cycles progress, the fitness function is adjusted by relaxing non-physical
fitness constraints such that the final round is typically pure energy
minimization. This process is repeated over many independent trials, producing a
large number of minimal energy sequence-structure pairs predicted to be fit for
desired purpose. When feasible, the most promising of these are subjected to a
more computationally intensive structure prediction technique called forward
folding (REF?) to better assess whether the designs are indeed globally minimum
energy given their sequence.

\subsection{Proposed Work for Design Phase 1}\label{proposed1}

We propose to dramatically improve Phase 1 of our design process through the use
of a seamless multi-scale model we call a rotamer interaction field (RIF). This
model employs hierarchy of bounding functions which estimate the best possible
fitness achievable within an ensemble of $\Phi^{3N}$ $\Re^6$ position of a given radius.
The
bounding functions use spatial indexing techniques (REF jorge paper) to rapidly
look up precomputed locally optimal solutions to small subproblems of the full
sequence/CHI optimization. The product of these sub-solutions is not generally a
feasible solution, but provides a bound on the maximum achievable fitness. In
preliminary work, we have applied these bounds to binder design using a nesting
hierarchical decomposition of $\Re^6$ space to perform a branch and bound search
called RIFdock. We are in the process of conducting an extensive computational
and experimental benchmark of RIFdock. By computational metrics, designs
produced with RIFdock are significantly better than those produced with three
other previous methods. RIFdock has also produced working binders to IL17 with
far better binding affinity than has ever been achieved with purely
computational methods (Franziska et. al., unpublished). We propose to
collaborate with the computer science department to improve RIF through the use
of tighter, more sophisticated bounds, such as those possible using cost
function networks (REF THOMAS PAPER), and to expand the RIFdock search technique
to $\Phi^{3N}$ as well as $\Re^6$.

\subsection{Proposed Work for Design Phase 2}\label{proposed2}

We propose to dramatically improve Phase 2 of our design process by expanding
the class of fitness functions that can be efficiently optimized using
combinatorial optimization techniques from computer science (REFs SAT, CFN, CST,
MIP etc). Such techniques have been explored for protein design by our group and
others (REFs thomas, ALF), but have not replaced the monte-carlo search used in
rosetta because rigorously optimizing the fitness function is not worth the
extra computational cost; designs must be exact energy minima for their
sequence, but fitness must only be good enough for the desired purpose. However,
the current monte carlo optimization technique is limited to objective functions
containing single and pairwise terms. This has proven sufficient to design
hydrophobic protein cores and interfaces, but cannot produce fully satisfied
polar networks like those observed in natural proteins. Such networks increase
the specificity and solubility of designed proteins, greatly increasing the
likelihood both that they are globally minimum energy for their sequence, and
that they will function in the designed way. We have developed a very successful
technique called HBNet (REF scotts paper) that produces native-like polar
networks using an extremely computationally demanding brute force search. In
collaboration with members of the UW computer science dept, we have recently
demonstrated that polar networks can be designed orders of magnitude faster than
brute force using any of a variety of classes of combinatorial solvers including
partial weighted MaxSAT (REF), mixed integer programming (REF), and cost
function networks (REF). Cost function networks are particularly promising, as
they are also the leading method for exact solution of the general sequence
optimization problem in Phase 2. We propose to further explore the use of these
solvers to design HBNets, and to integrate these solvers into Phase 2 of our
design process so that HBNet design, and other complex constraints, can be
incorporated directly into fitness functions used in protein design.


The above described advances in methodology will enhance blah blah.






\section{Quantitative​ ​ and​ ​ Interpretable​ ​ Machine​ ​ Learning}

The​ ​ computation​ ​ and​ ​ characterization​ ​ of​ ​ complex​ ​ and​ ​ multi-scale​ ​ biophysics​ ​ (CMSBP)
phenomenon​ ​ is​ ​ at​ ​ the​ ​ forefront​ ​ of​ ​ modern​ ​ mathematical​ ​ and​ ​ scientific​ ​ exploration.​ ​ Neuronal
networks​ ​ of​ ​ the​ ​ brain,​ ​ large-scale​ ​ biochemical​ ​ kinetics,​ ​ cell​ ​ network​ ​ dynamics,​ ​ and​ ​ gene
regulatory​ ​ networks,​ ​ for​ ​ instance,​ ​ remain​ ​ some​ ​ of​ ​ the​ ​ most​ ​ difficult​ ​ and​ ​ challenging​ ​ problems​ ​ for
accurate​ ​ quantification​ ​ and​ ​ future​ ​ state​ ​ prediction.​ ​ There​ ​ are​ ​ promising​ ​ indicators​ ​ that​ ​ these
high-dimensional,​ ​ multi-scale​ ​ problems​ ​ may​ ​ be​ ​ tractable​ ​ with​ ​ emerging​ ​ mathematical
techniques.​ ​ These​ ​ techniques​ ​ aim​ ​ to​ ​ exploit​ ​ the​ ​ dominant​ ​ low-rank​ ​ structures​ ​ (features)​ ​ that​ ​ are
typically​ ​ exhibited.​ ​ Indeed,​ ​ much​ ​ of​ ​ the​ ​ success​ ​ of​ ​ reduced​ ​ order​ ​ models​ ​ (ROMs)​ ​ rests​ ​ on​ ​ the
ability​ ​ to​ ​ identify​ ​ and​ ​ take​ ​ advantage​ ​ of​ ​ patterns​ ​ and​ ​ features​ ​ in​ ​ high-dimensional​ ​ data.​ ​ These
low-dimensional​ ​ patterns​ ​ are​ ​ often​ ​ identified​ ​ using​ ​ dimensionality​ ​ reduction​ ​ techniques [​ 1 ​ ] ​ ​ such
as​ ​ the​ ​ proper​ ​ orthogonal​ ​ decomposition​ ​ (POD) [​ 2 ​ , ​ ​ 3 ​ , ​ ​ 4 ​ ] ​ ​ or​ ​ more​ ​ recently​ ​ via​ ​ dynamic​ ​ mode
decomposition​ ​ (DMD) [​ 5 ​ , ​ ​ 6 ​ , ​ ​ 7 , ​ ​ ​ 8 ​ ].
In​ ​ addition​ ​ to​ ​ advances​ ​ in​ ​ dimensionality​ ​ reduction,​ ​ key​ ​ developments​ ​ in​ ​ optimization,
compression,​ ​ and​ ​ the​ ​ geometry​ ​ of​ ​ sparse​ ​ vectors​ ​ in​ ​ high-dimensional​ ​ spaces​ ​ are​ ​ providing
powerful​ ​ new​ ​ techniques​ ​ to​ ​ obtain​ ​ approximate​ ​ solutions​ ​ to​ ​ NP-hard,​ ​ combinatorially​ ​ difficult
problems​ ​ in​ ​ scaleable​ ​ convex​ ​ optimization​ ​ architectures [​ 9 ​ ].​ ​ For​ ​ example,​ ​ compressed
sensing [​ 10​ , ​ ​ 11​ , ​ ​ 12​ ] ​ ​ provides​ ​ convex​ ​ algorithms​ ​ to​ ​ solve​ ​ the​ ​ combinatorial​ ​ sparse​ ​ signal
reconstruction​ ​ problem​ ​ with​ ​ high​ ​ probability.​ ​ Ideas​ ​ from​ ​ compressed​ ​ sensing​ ​ have​ ​ been​ ​ used​ ​ to
determine​ ​ the​ ​ optimal​ ​ sensing​ ​ locations​ ​ for​ ​ categorical​ ​ decisions​ ​ based​ ​ on​ ​ high-dimensional
data [​ 13​ ],​ ​ including​ ​ in​ ​ fluid​ ​ dynamics [​ 14​ , ​ ​ 15​ ].​ ​ Recently,​ ​ compressed​ ​ sensing,
sparsity-promoting​ ​ algorithms​ ​ such​ ​ as​ ​ the​ ​ lasso ​ ​ regression [​ 16​ , ​ ​ 17​ , ​ ​ 18​ ],​ ​ and​ ​ machine​ ​ learning
have​ ​ been​ ​ increasingly​ ​ applied​ ​ to​ ​ characterize​ ​ and​ ​ control​ ​ dynamical​ ​ systems [​ 19​ , ​ ​ 20​ , ​ ​ 21​ , ​ ​ 22​ ,
23​ , ​ ​ 24​ , ​ ​ 25​ , ​ ​ 26​ , ​ ​ 27​ , ​ ​ 28​ , ​ ​ 29​ , ​ ​ 30​ , ​ ​ 31​ ].​ ​ These​ ​ techniques​ ​ have​ ​ been​ ​ effective​ ​ in​ ​ modeling​ ​ high
dimensional​ ​ fluid​ ​ systems​ ​ using​ ​ POD [​ 32​ ] ​ ​ and​ ​ DMD [​ 33​ , ​ ​ 34​ , ​ ​ 35​ , ​ ​ 36​ ].​ ​ Information​ ​ criteria [​ 37​ , ​ ​ 38​ ,
39​ ] ​ ​ has​ ​ also​ ​ been​ ​ leveraged​ ​ for​ ​ the​ ​ sparse​ ​ identification​ ​ of​ ​ nonlinear​ ​ dynamics [​ 40​ ],​ ​ as​ ​ in [​ 41​ ,
42​ ].
For​ ​ CMSBP,​ ​ these​ ​ advances​ ​ are​ ​ insufficient​ ​ for​ ​ providing​ ​ guidelines​ ​ for​ ​ building​ ​ accurate
ROMs.​ ​ Indeed,​ ​ if​ ​ the​ ​ various​ ​ multi-scale,​ ​ spatio-temporal​ ​ features​ ​ of​ ​ the​ ​ biological​ ​ data​ ​ are​ ​ not
separated,​ ​ then​ ​ an​ ​ artificially​ ​ high-rank​ ​ for​ ​ the​ ​ dynamics​ ​ is​ ​ exhibited.​ ​ In​ ​ contrast,​ ​ if​ ​ time​ ​ scales
are​ ​ separated​ ​ and​ ​ their​ ​ individual​ ​ low-rank​ ​ embeddings​ ​ constructed,​ ​ then​ ​ the​ ​ machine​ ​ learning
and​ ​ sparse​ ​ sampling​ ​ can​ ​ once​ ​ again​ ​ be​ ​ combined​ ​ to​ ​ great​ ​ effect.​ ​ The​ ​ multi-resolution​ ​ dynamic
mode​ ​ decomposition​ ​ (mrDMD) [​ 43​ ] ​ ​ is​ ​ an​ ​ ideal​ ​ tool​ ​ for​ ​ separating​ ​ multiscale​ ​ data,​ ​ thus​ ​ allowing
us​ ​ to​ ​ more​ ​ readily​ ​ capitalize​ ​ on​ ​ the​ ​ recent​ ​ innovations​ ​ in​ ​ ROMs​ ​ and​ ​ model​ ​ inference
techniques.​ ​ Thus​ ​ key​ ​ advances​ ​ in​ ​ a ​ ​ number​ ​ of​ ​ fields​ ​ are​ ​ fundamentally​ ​ changing​ ​ our​ ​ approach
to​ ​ the​ ​ acquisition,​ ​ analysis,​ ​ and​ ​ model​ ​ building​ ​ from​ ​ data​ ​ acquired​ ​ in​ ​ multi-scale​ ​ systems.
Specific​ ​ to​ ​ this​ ​ proposal​ ​ are​ ​ (i)​ ​ advances​ ​ in​ ​ DMD​ ​ which​ ​ can​ ​ be​ ​ used​ ​ to​ ​ disentangle​ ​ and​ ​ exploit
spatio-temporal​ ​ patterns​ ​ in​ ​ high-dimensional​ ​ data,​ ​ and​ ​ (ii)​ ​ model​ ​ identification​ ​ methods​ ​ which
can​ ​ be​ ​ used​ ​ to​ ​ construct​ ​ optimal,​ ​ data-driven​ ​ models​ ​ at​ ​ different​ ​ temporal​ ​ scales​ ​ through​ ​ DMD,
Koopman​ ​ theory​ ​ and/or​ ​ sparse​ ​ identification​ ​ of​ ​ nonlinear​ ​ dynamics [​ 44​ ].​ ​ Importantly,​ ​ the​ ​ DMD
architecture​ ​ allows​ ​ for​ ​ a ​ ​ mathematical​ ​ strategy​ ​ for​ ​ connecting​ ​ and/or​ ​ inferring​ ​ different
spatio-temporal​ ​ scales.
Reduced​ ​ order​ ​ models​ ​ (ROMs)​ ​ provide​ ​ a ​ ​ transformative​ ​ paradigm​ ​ for​ ​ simulating
high-dimensional​ ​ systems​ ​ of​ ​ equations​ ​ which​ ​ are​ ​ often​ ​ derived​ ​ from​ ​ discretizing​ ​ partial
differential​ ​ equations​ ​ (PDEs).​ ​ Specifically,​ ​ the​ ​ goal​ ​ of​ ​ ROMs​ ​ is​ ​ to​ ​ replace​ ​ simulations​ ​ of​ ​ an
n ​ -dimensional​ ​ ( ​ n ​ ≫ 1)​ ​ system​ ​ of​ ​ differential​ ​ equations​ \ ​ bfu ̇ = ​ F ​ (\bf​ u ​ , ​ μ ​ ) ​ ​ with​ ​ an​ ​ r ​ -dimensional
surrogate​ ​ model​ \ ​ bfa = ​ f ​ (\bf​ a ​ , ​ μ ​ ) ​ ​ where​ ​ r ​ ≪ ​ n ​ ​ and​ ​ the​ ​ parameter​ ​ vector​ ​ μ ​ ​ represents​ ​ an​ ​ explicit
parametric​ ​ dependence.​ ​ The​ ​ reduction​ ​ in​ ​ dimension​ ​ is​ ​ accomplished​ ​ by​ ​ projecting​ ​ to​ ​ an​ ​ optimalsubspace​ ​ \bfΨ,​ ​ the​ ​ proper​ ​ orthogonal​ ​ decomposition​ ​ (POD)​ ​ modes,​ ​ which​ ​ is​ ​ computed​ ​ from​ ​ a
singular​ ​ value​ ​ decomposition​ ​ of​ ​ snapshots​ ​ of​ ​ the​ ​ high-dimensional​ ​ system​ ​ \bf​ u ​ . ​ ​ Despite​ ​ the
many​ ​ successes​ ​ of​ ​ the​ ​ ROM​ ​ method,​ ​ critical​ ​ challenges​ ​ remain.​ ​ In​ ​ this​ ​ proposal,​ ​ we​ ​ address
two​ ​ of​ ​ these​ ​ open​ ​ challenges:​ ​ (i)​ ​ a ​ ​ data-driven​ ​ approach​ ​ for​ ​ discovering​ ​ the​ ​ governing
equations​ ​ F ​ (\bf​ u ​ , ​ μ ​ ) ​ ​ when​ ​ time-series​ ​ measurements​ ​ of​ ​ the​ ​ system​ ​ alone​ ​ are​ ​ available,​ ​ and​ ​ (ii)
an​ ​ efficient​ ​ method​ ​ for​ ​ building​ ​ ROMs​ ​ that​ ​ accounts​ ​ for​ ​ the​ ​ parametric​ ​ dependence​ ​ of​ ​ the
governing​ ​ equations​ ​ on​ ​ μ ​ ( t ​ ​ ).​ ​ By​ ​ integrating​ ​ sparse​ ​ sampling​ ​ methods​ ​ with​ ​ recent​ ​ regression
techniques​ ​ for​ ​ the​ ​ discovery​ ​ of​ ​ dynamical​ ​ systems,​ ​ an​ ​ online​ ​ and​ ​ non-intrusive​ ​ method​ ​ is
demonstrated​ ​ for​ ​ the​ ​ discovery​ ​ of​ ​ parametric​ ​ systems​ ​ which​ ​ is​ ​ capable​ ​ rapid​ ​ model​ ​ updates
without​ ​ recourse​ ​ to​ ​ the​ ​ full​ ​ high-dimensional​ ​ system.​ ​ Indeed,​ ​ low-rank​ ​ ROMs​ ​ are​ ​ identified​ ​ and
updated​ ​ directly​ ​ from​ ​ time​ ​ series​ ​ data​ ​ while​ ​ concurrently​ ​ inferring​ ​ the​ ​ governing​ ​ spatio-temporal
equations​ ​ responsible​ ​ for​ ​ generating​ ​ the​ ​ data.
The​ ​ traditional​ ​ ROM​ ​ architecture​ ​ assumes​ ​ that​ ​ the​ ​ underlying​ ​ evolution​ ​ dynamics
\bf u ̇ = ​ F ​ (\bf​ u ​ , ​ μ ​ ) ​ ​ is​ ​ known.​ ​ However,​ ​ in​ ​ a ​ ​ majority​ ​ of​ ​ applications​ ​ across​ ​ the​ ​ biological​ ​ sciences,
the​ ​ governing​ ​ equations​ ​ are​ ​ unknown​ ​ or​ ​ only​ ​ partially​ ​ known.​ ​ Or​ ​ alternatively,​ ​ observed
low-rank​ ​ dynamics​ ​ occurs​ ​ due​ ​ to​ ​ microscale​ ​ interactions​ ​ which​ ​ generate​ ​ emergent​ ​ macroscale
properties.​ ​ Thus​ ​ there​ ​ exists​ ​ a ​ ​ variety​ ​ of​ ​ white-,​ ​ gray-​ ​ and​ ​ black-box​ ​ modeling​ ​ strategies​ ​ for
dealing​ ​ with​ ​ full,​ ​ partial​ ​ or​ ​ no​ ​ knowledge​ ​ of​ ​ the​ ​ governing​ ​ system​ ​ respectively.​ ​ Emerging
data-driven​ ​ strategies​ ​ are​ ​ giving​ ​ rise​ ​ to​ ​ new​ ​ mathematical​ ​ architectures​ ​ that​ ​ can​ ​ learn​ ​ white-box
models​ ​ from​ ​ time​ ​ series​ ​ measurements​ ​ of​ ​ a ​ ​ given​ ​ system,​ ​ i.e.​ ​ the​ ​ correct​ ​ governing​ ​ model​ ​ can
be​ ​ discovered.​ ​ These​ ​ data-driven​ ​ methods,​ ​ which​ ​ are​ ​ rooted​ ​ in​ ​ sparse​ ​ regression,​ ​ have​ ​ been
proven​ ​ to​ ​ be​ ​ successful​ ​ for​ ​ discovering​ ​ a ​ ​ variety​ ​ of​ ​ differential​ ​ and​ ​ partial​ ​ differential​ ​ equations.
Further,​ ​ the​ ​ discovered​ ​ models​ ​ can​ ​ be​ ​ cross-validated​ ​ using​ ​ well​ ​ established​ ​ information
criteria​ ​ methods​ ​ from​ ​ model​ ​ selection.​ ​ In​ ​ the​ ​ innovations​ ​ to​ ​ be​ ​ developed​ ​ here,​ ​ we​ ​ discover​ ​ the
governing​ ​ low-dimensional​ ​ ROM​ ​ directly​ ​ from​ ​ data,​ ​ thus​ ​ bypassing​ ​ any​ ​ recourse​ ​ to​ ​ the
high-dimensional​ ​ system.​ ​ Moreover,​ ​ the​ ​ method​ ​ infers​ ​ the​ ​ governing​ ​ high-dimensional​ ​ PDE,
thus​ ​ providing​ ​ critical​ ​ insight​ ​ into​ ​ the​ ​ spatio-temporal​ ​ evolution​ ​ dynamics.​ ​ Specifically,​ ​ we
consider​ ​ a ​ ​ data-driven​ ​ method​ ​ whereby​ ​ low-rank​ ​ ROMs​ ​ can​ ​ be​ ​ directly​ ​ constructed​ ​ from​ ​ data
alone​ ​ using​ ​ SINDy​ ​ and​ ​ PDE-FIND.​ ​ The​ ​ methods​ ​ can​ ​ be​ ​ enacted​ ​ in​ ​ an​ ​ online​ ​ and​ ​ non-intrusive
fashion​ ​ using​ ​ randomized​ ​ (sparse)​ ​ sampling.​ ​ The​ ​ method​ ​ is​ ​ ideal​ ​ for​ ​ parametric​ ​ systems,
requiring​ ​ rapid​ ​ model​ ​ updates​ ​ without​ ​ recourse​ ​ to​ ​ the​ ​ full​ ​ high-dimensional​ ​ system.​ ​ Moreover,
the​ ​ method​ ​ discovers​ ​ the​ ​ high-dimensional,​ ​ white-box​ ​ PDE​ ​ model​ ​ responsible​ ​ for​ ​ generating
the​ ​ spatio-temporal​ ​ data.
Parametric​ ​ dependencies​ ​ of​ ​ the​ ​ underlying​ ​ PDE​ ​ compromise​ ​ not​ ​ only​ ​ the​ ​ PDE​ ​ and​ ​ ROM
discovery​ ​ process,​ ​ but​ ​ also​ ​ quickly​ ​ invalidates​ ​ a ​ ​ given​ ​ ROM.​ ​ For​ ​ such​ ​ systems,​ ​ the​ ​ low-rank
embedding​ ​ subspace​ ​ \bfΨ​ ​ requires​ ​ frequent​ ​ updating​ ​ since​ ​ the​ ​ ROM​ ​ model​ ​ is​ ​ based​ ​ upon
accurately​ ​ projecting​ ​ into​ ​ this​ ​ subspace​ ​ through​ ​ inner​ ​ products.​ ​ Recent​ ​ efforts​ ​ have​ ​ attempted
to​ ​ address​ ​ parametric​ ​ dependencies​ ​ by​ ​ the​ ​ construction​ ​ of​ ​ libraries​ ​ of​ ​ POD​ ​ modes​ ​ which​ ​ are
valid​ ​ across​ ​ a ​ ​ wide​ ​ range​ ​ of​ ​ parameters.​ ​ Thus​ ​ POD​ ​ modes​ ​ and​ ​ ROM​ ​ models​ ​ can​ ​ be​ ​ quickly
switched​ ​ out​ ​ as​ ​ appropriate​ ​ in​ ​ an​ ​ \em​ ​ online​ ​ manner.​ ​ In​ ​ the​ ​ method​ ​ proposed​ ​ here,​ ​ a ​ ​ new​ ​ ROM
can​ ​ be​ ​ quickly​ ​ computed​ ​ without​ ​ recourse​ ​ to​ ​ an​ ​ \em​ ​ offline​ ​ training​ ​ stage​ ​ (POD​ ​ libraries)​ ​ or
recourse​ ​ to​ ​ the​ ​ original​ ​ high-dimensional​ ​ system.​ ​ Moreover,​ ​ our​ ​ method​ ​ can​ ​ manage
parametric​ ​ dependencies​ ​ which​ ​ are​ ​ beyond​ ​ the​ ​ ability​ ​ of​ ​ current​ ​ methods​ ​ to​ ​ handle.​ ​ Figure ↓ ​
demonstrates​ ​ three​ ​ prototypical​ ​ parametric​ ​ dependencies:​ ​ (i)​ ​ a ​ ​ PDE​ ​ model​ \ ​ bfu ̇ = ​ F ​ (\bf​ u ​ ) ​ ​ whose
parameters​ ​ change​ ​ at​ ​ fixed​ ​ points​ ​ in​ ​ time,​ ​ (b)​ ​ a ​ ​ PDE​ ​ model​ ​ \bf u ̇ = ​ F ​ (\bf​ u ​ , ​ μ ​ ( ​ t ​ ))​ ​ that​ ​ depends
continuously​ ​ on​ ​ the​ ​ parameter​ ​ μ ​ ( ​ t ​ ),​ ​ and​ ​ (c)​ ​ a ​ ​ system​ ​ where​ ​ the​ ​ underlying​ ​ PDE​ ​ changes​ ​ in​ ​ time
from​ ​ \bf u ̇ = ​ F ​ (\bf​ u ​ ) ​ ​ to​ ​ \bf u ̇ = ​ G ​ (\bf​ u ​ ).​ ​ Although​ ​ methods​ ​ for​ ​ handling​ ​ the​ ​ first​ ​ case​ ​ have​ ​ been
developed,​ ​ (ii)​ ​ and​ ​ (iii)​ ​ remain​ ​ exceptionally​ ​ challenging.​ ​ ROM​ ​ discovery​ ​ provides​ ​ a ​ ​ principledapproach​ ​ to​ ​ efficiently​ ​ handling​ ​ all​ ​ three​ ​ parametric​ ​ cases​ ​ in​ ​ an​ ​ \em​ ​ online​ ​ fashion​ ​ without
recourse​ ​ to​ ​ the​ ​ high-dimensional​ ​ system.


\end{document}
